\documentclass[bigger]{beamer}
\usepackage[utf8]{inputenc}
\usepackage[T1]{fontenc}
\usepackage{color}
\usepackage{graphicx}
\usepackage{eurosym}
\usepackage{parcolumns}

\DeclareGraphicsExtensions{.png,.pdf,.jpg}
\graphicspath{{./pics/}}

%Global Background must be put in preamble
\usebackgroundtemplate%
{%
    \includegraphics[width=\paperwidth,height=\paperheight]{Design.jpg}%
}


\newcommand{\topic}[1]{{\huge{\textcolor{white}{\textbf{#1}}}}}
%\newcommand{\topic}[1]{\textbf{#1}}

\begin{document}
{
\usebackgroundtemplate{\includegraphics[width=\paperwidth]{Back.png}}%
\begin{frame}
%Title: leer, vielleicht noch unsere Namen einfügen
	\begin{figure}[H]
		\includegraphics[width=350px]{SprechblaseOR.png}
	\end{figure}

\end{frame}
}

\begin{frame}{\topic{Plant Guard}}
   \begin{itemize}
      \item Twitter für Kommunikation
      \item Messungen: Feuchtigkeit*, Temperatur, Wasserstand, uvm.
      \item kann Pflanzen gießen
      \item Erweiterbarkeit gegeben*
      \item Kosten sind relativ gering*
   \end{itemize}
*dazu später mehr Informationen
\end{frame}

\begin{frame}{\topic{Vergleich (andere Projekte)}}
	\begin{itemize}
		\item twittern oder nutzen spezielle Netzwerkprogramme
		\item meist keine Bewässerung (nur bei Hobbyprojekten)
		\item weniger Sensoren (meist nur Feuchtigkeit)
		\item dafür ein Projekt mit Pflanzen-DB für Konfiguration
		\item ein Bausatz/Gerät meist um {100\euro} oder nicht erhältlich
	\end{itemize}
\end{frame}

\begin{frame}{\topic{Vorteile des Plant Guard}}
	\begin{itemize}
		\item Pflanzeninfos über Twitter weltweit verfügbar
		%\item Multiuser (z.B. für die Gießvertretung)
		\item automatische Bewässerung
		\item erweiterbar (mehr Sensoren, mehr Pflanzen, ...)
		\item bei Stromausfall wird inzwischen später gegossen
	\end{itemize}
\end{frame}

\begin{frame}{\topic{Probleme}}
	\begin{itemize}
		\item Inkontinenz beim Start
		\item wenn der PlantGuard zuviel gießt, könnte der Topf überlaufen
		\item ziemlich großer Behälter, schlecht zu verstecken, leicht zu kippen, muss höher stehen als die Pflanze
		\item Abdichtung des Wasserbehälters
		\item viele und lange Kabel
		\item hohe Kosten bei verteilt stehenden Pflanzen (pro Pflanze ein PlantGuard)
	\end{itemize}
\end{frame}

\begin{frame}{\topic{Kosten}}
	\begin{itemize}
		\item Arduino+WiFly: ab 60\euro
		\item Ventil: 7,50\euro
		\item Sensoren: je nach Anzahl/Art
		\item Platine und Material: wenige \euro
	\end{itemize}
	Summe: ca. 75\euro
\end{frame}

\begin{frame}{\topic{Sensoren}}
	\begin{itemize}
		\item Temperatur
		\item Feuchtigkeit
		\item Wasserstand
	\end{itemize}
weitere Ideen folgen
\end{frame}

\begin{frame}{\topic{Temperatur}}
	\begin{itemize}
		\item benötigt einen OpAmp
		\item lange Drähte für freie Positionierung
		\item misst im Bereich der Raumtemperatur
		\item direkt auf der Platine
	\end{itemize}
\end{frame}

\begin{frame}{\topic{Temperatursensor}}
\includegraphics[width=0.4\linewidth]{temperatur.jpg}
\end{frame}

\begin{frame}{\topic{Feuchtigkeit}}
	\begin{itemize}
		\item zwei lange Metallstäbe, die mit festem Abstand zueinander in den Boden gesteckt werden
		\item Spannungsmessung mit der Erde als Widerstand. Je niedriger der Wert, desto feuchter die Erde
		\item Nutzung von Wechselspannung (durch Umpolen von 2 Pins) schont die Pflanze
	\end{itemize}
\end{frame}

\begin{frame}{\topic{Feuchtigkeitssensor}}
\begin{center}\includegraphics[width=\linewidth]{feuchtigkeit.jpg}\end{center}
\end{frame}

\begin{frame}{\topic{Wasserstand}}
	\begin{itemize}
		\item zwei sehr lange Drähte mit festem Abstand
		\item am Arduino werden 4,5 bis 4,9 Volt gemessen
		\item ca. 82stufige Messungen
		\item bedingt wichtig für Nachrichten
		\item sehr wichtig für Pumpen (nur im Wasser betreiben!)
	\end{itemize}
\end{frame}

\begin{frame}{\topic{Wasserstandssensor}}
\includegraphics[width=\linewidth]{wasser.jpg}
\end{frame}

\begin{frame}{\topic{Aktuatoren/Ventil}}
	\begin{itemize}
		\item leise
		\item nutzt Schwerkraft
		\item z.Z. auf Flaschen-Tank begrenzt, generell ist der Tank ersetzbar
		\item man muss das Ventil richtig ausrichten (um nicht den Bereich um die Pflanze zu gießen)
	\end{itemize}
\end{frame}

\begin{frame}
%Bild von aktueller Platine
\includegraphics[width=\linewidth]{board_final.jpg}
\end{frame}

\begin{frame}
%Bild von aktueller Platine
\includegraphics[width=\linewidth]{board_plant.jpg}
\end{frame}

\begin{frame}
%Bild von aktueller Platine
\includegraphics[width=\linewidth]{hardware.jpg}
\end{frame}

\begin{frame}{\topic{Bewässerung}}
% %früher / heute vergleich
%     \begin{columns}
%       \column[c]{.50\linewidth}
%         \begin{enumerate}
% 			\item verbreiterte Stellfläche
% 			\item Messkabel
% 			\item Ventilansteuerung
% 			\item Verbindung Vorrat $\rightarrow$ Ventil
%         \end{enumerate}
%       \column[c]{.50\linewidth}
%         \includegraphics[width=150px]{System.jpg}
%     \end{columns}
\url{http://www.youtube.com/watch?feature=player_embedded&v=Uci__lRXZsU}
\end{frame} 

\begin{frame}{\topic{Softwareübersicht}}
	\begin{itemize}
		\item 20kb groß (viel für Twitter)
		\item Regelschleife (wenn zu trocken, gieße)
		\item Zeitsteuerung mit "Gieße alle X Minuten" möglich
      \item Wichtige Parameter (z.B. Temperatur- und Feuchteschwellwerte) sind in eine Konfigurationsdatei ausgelagert
		\item liegt unter \url{http://github.com/jasinai/plant\_guard}
	\end{itemize}
\end{frame}

% \begin{frame}{\topic{Bewässerung}}
% %früher / heute vergleich
%        \includegraphics[width=350px]{Anschluss.jpg}
% \end{frame}

\begin{frame}{\topic{Tweets}}
	\begin{itemize}
      \item "Mir ist kalt! Stell mich an einen wärmeren Ort. Aktuelle Temperatur: X"
      \item "Puh, ist das warm! Ein bisschen Schatten wäre nicht schlecht. Aktuelle Temperatur: 33"
      \item "Mir ist warm. Es ist sind 30 C. Stell mich woanders hin."
      \item "Mensch, gib mir Wasser!"
	\end{itemize}
\end{frame}

\begin{frame}{\topic{mögliche Upgrades}}
	\begin{itemize}
		\item Pumpe statt Ventil (+{5\euro}, mehr Strom nötig, etwas lauter)
		\item mehr Pflanzen pro PlantGuard überwachen
		\item mehr Sensoren
		\item weg vom Arduino/WiFly
		\item Datenbank für Gießparameter
		\item diverse Hardwareverbesserungen
	\end{itemize}
\end{frame}

\begin{frame}{\topic{Pumpe statt Ventil}}
	\begin{itemize}
		\item Ventilansteuerung kompatibel mit einigen Pumpen
		\item Zeiten müssten angepasst werden
		\item Füllstand wäre noch wichtiger
		\item Behälter wäre beliebig, leichter zu verstecken
		\item kein Leck abzudichten, dafür eventuell mehr Schlauch zu verlegen
	\end{itemize}
\end{frame}

\begin{frame}{\topic{mehrere Pflanzen pro PlantGuard}}
	\begin{itemize}
		\item drückt die Kosten
		\item spart Strom (gegenüber einem Arduino pro Pflanze)
		\item mehr Anschlüsse nötig
    	\item Pflanzen sollten dicht beieinander stehen und möglichst selten bewegt werden
		\item DeMux: lange Kabel
		\item Funkchip: Protokoll ausarbeiten
	\end{itemize}
\end{frame}

\begin{frame}{\topic{weitere/andere Sensoren}}
	\begin{itemize}
		\item mehrere Feuchtigkeitssensoren
		\item längere Feuchtigkeitssensoren
		\item andere Füllstandssensorik
		\item mehrere Temperatursensoren
      	\item +Helligkeitssensor
		\item +Farbsensor, Kunstlicht/Sonne
		\item +Barometer
		\item +Kippsensor (für Notfälle)
	\end{itemize}
\end{frame}

% \begin{frame}{\topic{drahtlose Verbindung}}
% 	\begin{itemize}
% 		\item man spart sich eine (De)Mux-Platine für mehrere Messstationen
% 		\item man kann durch Anpassungen am Protokoll mehr Pflanzen versorgen
% 		\item muss gesichert werden (Übertragungsfehler, ...)
% 		\item keine Kabel quer durch den Raum/die Etage nötig
% 	\end{itemize}
% \end{frame}

% \begin{frame}{\topic{ohne Arduino}}
% 	\begin{itemize}
% 		\item NetIO mit Ethersex + SD-Karte = Webserver
% 		\item geringere Kosten, ca. 25\euro
% 		\item lange erprobt, sehr ausfallfrei
% 		\item sicher, da kein WLan benötigt wird
% 		\item kompatibel mit einem Sender-Chip
% 	\end{itemize}
% \end{frame}

\begin{frame}{\topic{Hardwareverbesserungen}}
	\begin{itemize}
		\item Hülle/Gehäuse
		\item professionelle Platine, eigenes Layout
		%\item Vorgeben zweier (De)mux-Schaltung (Norm für die Software)
	\end{itemize}
\end{frame}

\begin{frame}{\topic{Datenbank mit Pflanzendaten}}
	\begin{itemize}
		\item man müsste den PlantGuard nicht anlernen
		\item dafür aber eine Datenbank mit Informationen füllen und pflegen
		\item Datenbank liefert eine passende config.h
		\item Upload über die Arduino-IDE bzw. avrdude
		\item Nachteil: User könnte die falsche Pflanze wählen oder in der Größe irren
	\end{itemize}
\end{frame}

\usebackgroundtemplate{\includegraphics[width=\paperwidth]{Back.png}}%
\begin{frame}{}
	\begin{figure}[H]
		\includegraphics[width=350px]{Danke.png}
	\end{figure}
\end{frame}


\end{document}
